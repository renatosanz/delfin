\documentclass[12pt]{article}
\usepackage[spanish]{babel}
\usepackage[T1]{fontenc}
\usepackage{inter}
\usepackage{sectsty}
\usepackage{tgpagella}

\allsectionsfont{\sffamily}

\title{\sffamily \textbf{Plan de trabajo / Estancia Delfin}}
\date{\today}
\author{\sffamily Renato Sánchez Loeza}

\begin{document}
\maketitle

\section*{Semana 1}
\subsection*{Observaciones y Panorama general de los métodos de obtención de datos}
Introducción a los temas relacionados con la observación del firmamento y la historia detrás de este tema. Desde la observación con el ojo humano, pasando por la fotometría, los primeros CCD hasta llegar a las técnicas más actuales como el barrido con rendijas, el interferómetro de Fabry-Pérot (Etalon) y la espectroscopía multiobjetos (MOS).

\subsection*{Introducción a la espectroscopía integral de campo}
Revisión de la técnica de espectroscopía integral de campo o Integral Field Spectroscopy (IFS) y sus diferentes unidades integrales de campo o Integral Field Units (IFU): Arreglos de lentes, Arreglos de fibras y Cortadores/rebanadores de imagenes. De los cuales se obtienen diferentes resultados en la práctica.

\section*{Semana 2}
\subsection*{Descarga de Datos astronómicos}
Descarga de los cubos de datos procedentes del MaNGA/Marvin. Además de la exploración del archivo .fits el cual es utilizado para el resguardo de los datos astronómicos.

\subsection*{Instalación y manejo de librerías de análisis de datos}
Instalación de las librerías necesarias para el procesamiento de los datos como lo son numpy, matplotlib y astropy. Estas anteriores serán utilizadas en un programa en Python para la visualización y análisis de los cubos de datos que podemos obtener de MaNGA.

\section*{Semana 3}
\subsection*{Selección de la muestra de cubos de datos}
Durante esta semana se realizará la selección de algunos cubos de datos con los que trabajaremos lo que resta de la instancia. Los resultados obtenidos serán en base a estas muestras.

\section*{Semana 4}
\subsection*{Medición y Identificación de Líneas de emisión}
Uso de técnicas de espectroscopía para medir e identificar las líneas de emisión en los cubos de datos seleccionados. Se analizarán las características espectrales para determinar propiedades físicas de las fuentes astronómicas.

\section*{Semana 5}
\subsection*{Observación de parámetros físicos y secundarios}
Determinación de parámetros físicos como la temperatura, densidad y velocidad de las fuentes. Además, se observarán parámetros secundarios que puedan proporcionar información adicional sobre los objetos estudiados.

\subsection*{Descripción del manejo de datos}
Redacción del manejo de datos en el análisis de las semanas previas con el fin de aclarar el método de procesamiento al cual fueron sometidos los cubos de datos. Al mismo tiempo, se documentará el proceso de investigación.

\subsection*{Abundancias químicas y Poblaciones estelares}
Análisis de las abundancias químicas presentes en los datos y estudio de las poblaciones estelares. Identificación de diferentes elementos y su distribución para entender mejor la evolución de las galaxias.

\section*{Semana 6}
\subsection*{Producción de imágenes y rutinas de visualización}
Conclusión del sistema que permite la visualización y análisis de datos procedentes de los cubos de datos de forma que se puedan utilizar para deducir teorías acerca de las características de la galaxia en estudio. Además, producción de imágenes que reflejen dicha información.

\subsection*{Crear mapas de características físicas o derivadas}
Creación de mapas que representen las características físicas o derivadas de los objetos estudiados, como distribución de velocidades, temperatura y composición química de las galaxias.

\section*{Semana 7}
\subsection*{Producción del Documento Final y Poster}
Compilación y redacción del documento final, incluyendo todos los resultados y análisis realizados, con el fin de documentar lo realizado durante la instancia. Además de un poster científico con los mismo resultados.

\end{document}
